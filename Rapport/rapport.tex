%-----------------------------------------------------------------------
%	PACKAGES AND OTHER DOCUMENT CONFIGURATIONS
%-----------------------------------------------------------------------

\documentclass[a4paper, 11pt, ]{article} % Font size (can be 10pt, 11pt or 12pt) and paper size (remove a4paper for US letter paper)

%%%%%%%%%%%%%%%%%%%%%%%%%%%%
% TYPOGRAPHY
\usepackage[utf8]{luainputenc}
%\usepackage[utf8x]{inputenc} % Encodage
%\usepackage{mathpazo} % Use the Palatino font
\usepackage[T1]{fontenc} % Required for accented characters
%\linespread{1.05} % Change line spacing here, Palatino benefits from a slight increase by default
%\usepackage[protrusion=true,expansion=true]{microtype} % Better typography

%%%%%%%%%%%%%%%%%%%%%%%%%%%%
% LANGAGE
\usepackage{polyglossia}
\setmainlanguage{french}

%\makeatletter
%\renewcommand\@biblabel[1]{\textbf{#1.}} % Change the square brackets for each bibliography item from '[1]' to '1.'
%\renewcommand{\@listI}{\itemsep=0pt} % Reduce the space between items in the itemize and enumerate environments and the bibliography

%%%%%%%%%%%%%%%%%%%%%%%%%%%%
% HYPERREF
\usepackage{hyperref} % Links managment
\usepackage{url}
\hypersetup{
  colorlinks=true,
  citecolor=blue,
  urlcolor=blue,
  linktoc=all,
  linkcolor=black,
  plainpages=false,
  bookmarksnumbered,
}
% For print version, use this instead:
%\usepackage[pdfusetitle,bookmarksnumbered,plainpages=false]{hyperref}
%\usepackage{backref}
%\renewcommand{\backrefpagesname}{Cited on}

%%%%%%%%%%%%%%%%%%%%%%%%%%%%
% COLORS
\usepackage[dvipsnames]{xcolor}

%%%%%%%%%%%%%%%%%%%%%%%%%%%%
% GRAPHICS
\usepackage{graphicx} % Required for including pictures
\usepackage{epstopdf}
\usepackage{wrapfig} % Allows in-line images

%%%%%%%%%%%%%%%%%%%%%%%%%%%%
% PAGE
\usepackage{geometry} % Margin configuration
\geometry{hmargin=2.5cm,vmargin=1.5cm}
\usepackage{multicol} % Columns managment

%%%%%%%%%%%%%%%%%%%%%%%%%%%%
% TIKZ
\usepackage{tikz} %TIKZ package required for draw graphics
\usetikzlibrary{shapes,arrows}
\usepackage{pgfplots}
\usepackage{pgfopts}

\pgfplotsset{
  compat=newest,
  xlabel near ticks,
  ylabel near ticks
}

\usepackage{booktabs}

%%%%%%%%%%%%%%%%%%%%%%%%%%%%
% UML (tikz)
\usepackage{tikz-uml}
\usepackage{ifthen} %UML
\usepackage{xstring}
\usepackage{calc}

%%%%%%%%%%%%%%%%%%%%%%%%%%%%
% CODING STYLE
\input{template/listings} % for coding
\renewcommand{\lstlistlistingname}{Table des codes source}

%%%%%%%%%%%%%%%%%%%%%%%%%%%%
% TABLE OF CONTENT STYLE
%\usepackage[nottoc, notlof, notlot]{tocbibind}
\makeatletter
\renewcommand*\l@section[2]{%
  \ifnum \c@tocdepth >\m@ne
    \addpenalty{-\@highpenalty}%
    \vskip 1.0em \@plus\p@
    \setlength\@tempdima{1.5em}%
    \begingroup
      \parindent \z@ \rightskip \@pnumwidth
      \parfillskip -\@pnumwidth
      \leavevmode \bfseries
      \advance\leftskip\@tempdima
      \hskip -\leftskip
      #1\nobreak\ 
       \leaders\hbox{$\m@th
        \mkern \@dotsep mu\hbox{.}\mkern \@dotsep
        mu$}\hfil\nobreak\hb@xt@\@pnumwidth{\hss #2}\par
      \penalty\@highpenalty
    \endgroup
  \fi}
\makeatother

%%%%%%%%%%%%%%%%%%%%%%%%%%%%
% TITLE STYLE
\makeatletter
\renewcommand\@biblabel[1]{\textbf{#1.}} % Change the square brackets for each bibliography item from '[1]' to '1.'
\renewcommand{\@listI}{\itemsep=0pt} % Reduce the space between items in the itemize and enumerate environments and the bibliography

\renewcommand{\maketitle}{ % Customize the title - do not edit title and author name here, see the TITLE block below
\begin{flushright} % Right align
{\Huge\@title} % Increase the font size of the title

\vspace{50pt} % Some vertical space between the title and author name

{\Large\@author} % Author name
\\\Large\@date % Date

\vspace{40pt} % Some vertical space between the author block and abstract
\end{flushright}
}
\title{\textbf{Simulation multi-agents}\\
{\Large Simulation de \emph{Alien Versus Predator}}} % Subtitle

\author{\textsc{Valentin Laurent - Valentin Elvassore} % Author
\\{\textit{ISIMA F2 Simulation/Génie logiciel}}} % Institution

\date{15 Mars 2014} % Date

%----------------------------------------------------------------------------------------

%%%%%%%%%%%%%%%%%%%%%%%%%%%%
\begin{document}

\maketitle % Print the title section

%%%%%%%%%%%%%%%%%%%%%%%%%%%%
% BODY
%%%%%%%%%%%%%%%%%%%%%%%%%%%%

\tableofcontents
\newpage

\section*{Introduction}
\addcontentsline{toc}{section}{Introduction}
Un systeme multi-agents est un système composé d'un ensemble d'agents situé dans un environnement précis et interragissant selon certaines relations. Chaque agent est autonome vis-à-vis des autres agents et seul son modèle d'intelligence artificielle et la gestion des nombres aléatoires pourra influencer son comportement. 

Dans le cadre des simulations multi-agents nous avons choisis de faire une simulation de vie entre des Aliens, des Prédators et des Humains. Chaque entité ayant des besoins différents, il nous a paru interressant de faire cette simulation en nous concentrant sur les spécificités de chaque agent.
\cleardoublepage

\section{Contexte du projet}
Le projet de simulation Multi-Agents s'est porté sur le choix d'une adaptation d'\emph{Alien Versus Predator} car il permet d'introduire des règles complexes au sein d'un environnement assez simple. Tout l'intêret du projet est la définition des règles déterministes du jeu pour chacun des agents afin de le faire évouluer à partir d'un état initial.

\subsection{Conception des règles}

Nous avons voulu commencer par des règles simples que nous voulions complexifier par la suite grâce à des ajouts dans le code de contraintes pour chaque Agent. Les règles simples qui constituent la base du code sont les suivantes :
\begin{enumerate}
  \item Les Aliens tuent les Humains.
  \item Les Humains tuent les Prédators.
  \item Les Prédators tuent les Aliens.
\end{enumerate}

Ces règles permettent un certain équilibre dans le programme et chaque entité a ses chances de gagner la battaille pour la survie de sa race.

À cela j'ai ajouté un certain nombre de règle supplémentaire pour chaque agent :
\begin{itemize}
  \item Chaque agent vieillit à chaque tour et peut mourir de vieillesse à partir du 30\textsuperscript{ième} tour.
  \item Chaque agent a une majorité sexuelle fixé à 15 ans, \textit{c'est-à-dire} qu'il ne peut se reproduire qu'à partir de cet âge.
  \item Chaque agent a un champs de vision : chacun de leur déplacement sera fait en fonction de ce qu'il observe autour de lui. Le champ de vision amène ici un déséquilibre entre les entités puisque le Prédator a un champ de vision fixé à 6 cases tandis que les humains voient à 3 cases et les Aliens à 4 cases.
  \item Chaque entité a des points de vie afin qu'elle puisse survivre aux attaques et prévoir certaines interractions lorsqu'un agent est blessé. Cependant, pour les besoins du deboguage et par soucis de simplification du code chaque attaque tue directement la cible de cette attaque.
\end{itemize}

\subsection{Principe de la simulation}
Le terrain de jeu est une matrice déterminée selon les paramètres d'entrée du programme. C'est avec ces paramètres que cette matrice est crée, selon les types des Agents, cette création se fait via la methode :

\begin{lstlisting}[style=c++]
/**
 * Initialize the matrix of Agents.
 */
void DataStorage::initMatrix()
{
	double random;
	_matrix = new Agent**[_width];

	for (int i = 0 ; i < _width ; ++i)
	{
		_matrix[i] = new Agent*[_height];
		for (int j = 0 ; j < _height ; ++j)
		{
			random = GameController::getInstance()->getRandom();
			if (random < PROPORTIONS_CUMULEES[0])
            {
				_matrix[i][j] = new Alien(i,j);
            }
			else if (random < PROPORTIONS_CUMULEES[1])
            {
				_matrix[i][j] = new Human(i,j);
            }
			else if (random < PROPORTIONS_CUMULEES[2])
            {
				_matrix[i][j] = new Predator(i,j);
            }
			else
            {
				_matrix[i][j] = NULL;
            }
            
			if (_matrix[i][j])
            {
                _agents.push_back(_matrix[i][j]);
            }
		}
	}
}
\end{lstlisting}

La création du terrain de jeu se fait via le tirage d'un nombre aléatoire selon une table de répartition. Selon cette table de répartition et le tirage de nombre aléatoire entre 1 et 0 on peut déterminer si la case du terrain de jeu est un Agent (Alien, Prédator et Human) ou si c'est une case vide.
\begin{figure}[h]
\centering
\begin{multicols}{2}
\begin{tikzpicture}[font=\small]
    \begin{axis}[
      ybar,
      bar width=25pt,
      xlabel={\bfseries Agents},
      ylabel={\bfseries Probabilité},
      ymin=0,
      ytick=\empty,
      xtick=data,
      axis x line=bottom,
      axis y line=left,
      enlarge x limits=0.2,
      symbolic x coords={alien,predator,human,vide},
      xticklabel style={anchor=base,yshift=-\baselineskip},
      nodes near coords={\pgfmathprintnumber\pgfplotspointmeta\%}
    ]
      \addplot[fill=MidnightBlue!40] coordinates {
        (alien,4)
        (predator,4)
        (human,4)
        (vide,88)
      };
    \end{axis}
  \end{tikzpicture}

\begin{tabular}{c}
  \textit{Table cumulant} \\
  \textit{les répartitions} \\
  \hline
  \hline
  1 \\
  0,12 \\
  0,08 \\
  0,04 \\ \hline
\end{tabular}

\end{multicols}
\caption{Distribution des agents dans la matrice de jeu.}
\end{figure}

\newpage
Le même principe est appliqué pour la création des comportements de chaque Agent. Un nombre aléatoire est tiré et selon une table des répartition que j'ai fixé arbitrairement afin d'équilibrer le déroulement de la simulation, chaque agent se voit attribuer un comportement qui déterminera ses actions au cours du tour de la simulation.

\begin{figure}[h]
\centering
\begin{multicols}{2}
\begin{tikzpicture}[font=\small]
    \begin{axis}[
      ybar,
      bar width=25pt,
      xlabel={\bfseries Comprtement},
      ylabel={\bfseries Probabilité},
      ymin=0,
      ytick=\empty,
      xtick=data,
      axis x line=bottom,
      axis y line=left,
      enlarge x limits=0.2,
      symbolic x coords={agressif,defensif,en rut},
      xticklabel style={anchor=base,yshift=-\baselineskip},
      nodes near coords={\pgfmathprintnumber\pgfplotspointmeta\%}
    ]
      \addplot[fill=OrangeRed!50] coordinates {
        (agressif,40)
        (defensif,30)
        (en rut,30)
      };
    \end{axis}
  \end{tikzpicture}

\begin{tabular}{c}
  \textit{Table cumulant} \\
  \textit{les répartitions} \\
  \hline
  \hline
  1 \\
  0,7 \\
  0,3 \\ \hline
\end{tabular}

\end{multicols}
\caption{Distribution des actions pour chaque agent.}
\end{figure}
\cleardoublepage

\section{Conception de la solution}
\subsection{Présentation de la structure}
Nous avons choisis d'utiliser une structure DataLayer/BuisinessLayer/PresentationLayer avec des Entities accessible par les trois classes afin de structurer au mieux notre programme. Ainsi les classes \verb$DataStorage$, \verb$GameControler$ et \verb$ConsolePresentation$ suivent ce schéma avec une disposition de la classe \verb$Agent$ pour toutes ces classes.

\begin{figure}[h!]
	\centering
  	\begin{tikzpicture}
	\begin{umlpackage}[fill=White]{Presentation}
		\umlclass[y=-7,x=3,fill=White]{Console}{}{}
	\end{umlpackage}

	\begin{umlpackage}[fill=White]{Logic}
		\umlclass[fill=White, y=-2]{GameController}{}{}
		\umlclass[x=3, y=-4, fill=White]{MersenneTwister}{}{}
	\end{umlpackage}

	\begin{umlpackage}[fill=White]{Data}
		\umlclass[y=4, fill=White]{DataStorage}{}{}
	\end{umlpackage}

	\begin{umlpackage}[fill=White]{Entity}
		\umlclass[x=8, y=5, fill=White]{Agent}{}{}
		\umlclass[x=5.5, y=3, fill=White]{Alien}{}{}
		\umlclass[x=8, y=3, fill=White]{Human}{}{}
		\umlclass[x=10.5, y=3, fill=White]{Predator}{}{}
	\end{umlpackage}

	\begin{umlpackage}[fill=White]{Behaviour}
		\umlinterface[x=10, y=-2, fill=White]{IBehaviour}{}{}
		\umlclass[x=7.2, y=-5, fill=White]{Aggressive}{}{}
		\umlclass[x=10, y=-5, fill=White]{Defensive}{}{}
		\umlclass[x=12.7, y=-5, fill=White]{Courtship}{}{}
		\umlclass[x=13, y=-2, fill=White]{Movement}{}{}
	\end{umlpackage}

	\umlcompo[mult1=1, pos1=0.1, mult2=1, pos2=0.9, angle1=80, angle2=10, loopsize=2cm]{GameController}{GameController}
	\umluniassoc[geometry=--,arg=create(), pos=0.5]{DataStorage}{Entity}
	\umluniassoc[geometry=|-|, arg=changeData(), pos=1.5, anchor1=120, anchor2=-60]{Behaviour}{Data}

	\umlcompo[mult1=1, pos1=0.2, mult2=1, pos2=1.9, geometry=|-]{MersenneTwister}{GameController}

	\umlinherit[geometry=|-|]{Alien}{Agent}
	\umlinherit[]{Human}{Agent}
	\umlinherit[geometry=|-|]{Predator}{Agent}

	\umlaggreg[mult1=1, pos1=0.2, mult2=1, pos2=1.9, geometry=-|]{Console}{GameController}
	\umlaggreg[mult1=1, pos1=0.1, mult2=1, pos2=0.9,]{DataStorage}{GameController}
	\umlaggreg[mult1=1, pos1=0.1, mult2=1, pos2=0.9]{IBehaviour}{Movement}

	\umlinherit[geometry=|-|]{Aggressive}{IBehaviour}
	\umlinherit[]{Defensive}{IBehaviour}
	\umlinherit[geometry=|-|]{Courtship}{IBehaviour}

	\end{tikzpicture}
	\caption{Diagramme de classes de l'analyse de la \textit{Black Box}}
\end{figure}

De plus, le comportement des Agents est défini via un patron stratégie qui sépare les comportements \verb$AggressiveBehaviour$, \verb$DefensiveBehaviour$ et \verb$CourtshipBehaviour$. Ces trois classes sont définies à partir de la classe abstraite \verb$IBehaviour$ que j'ai mis en place comme une interface.

Avec plus de temps, je voulais mettre en place trois autres patrons stratégie pour chacun de ces comportements afin qu'ils soient spécifiques à chaque Agent et ainsi pouvoir implémenter facilement des règles plus subtiles pour chacun des Agents. Par exemple, un Alien ne se reproduit pas avec un autre Alien mais pond un parasite qui se développera dans sa victime, ou un Prédator est très mauvais au corps à corps et ne peut attaquer ces cibles qu'à distance.

La classe \verb$Mouvement$ permet de simplifier la gestion des déplacements. Un mouvement est instancié à chaque attribution de comportement. Chaque Agent va se déplacer selon son comportement, si ce mouvement permet une action l'Agent va agir en conséquence.

Les Agents n'ont pas une gestion à proprement dites "selon un patron stratégie", la classe abstraite Agent permet de définir toutes les méthodes et attributs communs à chaque Agent puis chaque Agent spécifique redéfinit ces méthodes selon ces caractéristiques propres.

\subsection{Utilisation de la librairie Boost}
Toute les classes de conportement \verb$Behaviour$ et la classe \verb$ouvement$ imposent un changement de la matrice originale en suivant les principes basés sur le jeu de la vie : toutes les actions qu'entreprennent chaque agent sont effectuées à partir de la copie de la matrice de jeu mais chaque action executée est répercuptée sur la matrice originale.

Le soucis est que cette matrice originale est gérée uniquement par la classe \verb$DataStorage$. Cette dernière est instanciée dans le \verb$GameControler$ qui suit un patron singleton afin de préserver l'unicité de cette structure. Les classes de type \verb$Behaviour$ sont indépendantes de cette structure Data/Buisiness/Presentation et sont instanciées et détrutes de multiple fois pendant un tour. Par soucis de simplicité et de lisibilité du code et de la structure, je ne voulais pas donner la \verb$DataStorage$ en argument des classes \verb$Behaviour$ afin de gérer la matrice originale à chaque action.

L'idée a été de créer des évènements : la \verb$DataStorage$ est abonnée aux évennements déclanchés par les \verb$Behaviour$s et la matrice est modifiée dès qu'un évennement est lancé et receptionné. Cette gestion des évènements s'est faites via les librairies Boost \verb$signal2$ et \verb$bind$. Les classes \verb$DataStorage$ et \verb$IBehaviour$ sont connectées dans le \verb$GameControler$ en précisant les méthodes.

\begin{lstlisting}[style=c++]
CourtshipBehaviour * lover = new CourtshipBehaviour();
currentAgent->changeBehaviour(lover);
lover->loveSig.connect(boost::bind(&DataStorage::giveBirth,&_data,_1,_2,_3));
lover->getMovement().addMoveSignal(boost::bind(&DataStorage::moveAgent,&_data,_1,_2,_3));
\end{lstlisting}

Ainsi lorsque deux Agents se reproduisent un signal est lancé (\verb$loveSig$) et la methode \verb$DataStorage::giveBirth$ est lancée avec les arguments \verb$_1$, \verb$_2$ et \verb$_3$.

Le signal est initialisé directement dans la classe de comportement \verb$CourtshipBehaviour$ et est lancé par la methode \verb$CourtshipBehaviour::act()$.

\begin{lstlisting}[style=c++]
#include <iostream>
#include <boost/signals2.hpp>
#include "IBehaviour.h"

/**
 * The courtship behaviour for each agent.
 * @class CourtshipBehaviour
 * @author Valentin Laurent
 */
class CourtshipBehaviour : public IBehaviour
{
public:
    //Constructor
    CourtshipBehaviour();
    
    //Destructor
    ~CourtshipBehaviour(){};    /**<Courtship behaviour destructor.*/
    
    //Signal
    boost::signals2::signal<void (Agent *, int, int)> loveSig;  /**<A signal to the DataStorage in order to give a birth to a new agent.*/
    
    //Methods
    bool love(Agent *, Agent ***, int, int, int);
    Movement::move_t findLove(Agent*, Agent***, int, int);
    void makeLove(Agent *, int, int);
    bool act(Agent *, Agent ***, int, int);
    
};
\end{lstlisting}


\cleardoublepage

\section{Présentation des résultats}
L'execution du programme demande de nombreux paramètres : le nombre de tours souhaités, la taille de la matrice de jeu et les options d'affichage.

L'execution \verb$alienvspredator 100 10 10 y$ donne l'initialisation de la matrice suivante avec un equilibrage entre le nombre d'Humains, d'Aliens et de Prédators :

\begin{verbatim}
P	.	.	.	A	.	.	.	.	.
.	.	H	P	.	.	.	.	.	.
.	.	.	P	.	.	.	.	.	.
.	.	.	.	.	.	.	H	.	.
.	.	.	A	H	.	.	.	.	.
.	.	.	.	.	.	.	.	.	.
.	.	P	.	.	.	.	.	.	.
.	.	.	.	.	.	.	.	.	H
.	A	.	.	.	P	.	.	.	.
.	.	.	A	.	.	P	.	.	H

Number of agents : 15
Number of deaths : 0
Number of births : 0
Number of iterations : 0
\end{verbatim}

Au bout de 100\textsuperscript{ième} tour on a :

\begin{verbatim}
.	.	.	.	.	.	.	.	.	.
.	.	.	.	.	.	.	.	.	.
.	.	.	.	.	.	.	.	.	.
.	.	.	.	.	.	.	.	.	.
.	.	.	.	.	A	A	.	H	.
.	A	.	.	A	.	.	A	A	.
.	.	A	.	A	A	A	A	A	A
.	.	.	.	.	A	A	A	A	A
.	.	.	A	A	A	A	A	A	A
.	.	A	A	A	A	.	A	A	A

Number of agents : 31
Number of deaths : 39
Number of births : 55
Number of iterations : 100
\end{verbatim}

On repère que tous les Prédators se sont fait tuer par les humains ou par vieillesse. De plus les Aliens dominent largement la partie, ceci est dû au fait que les agents on tendance à se bloquer dans un coin et à n'avoir que la possibilité de se reproduire avec les agents les plus proche.

On peut observer la mort des Prédators au tour 31 :

\begin{verbatim}
.	.	.	.	.	.	.	.	.	.
.	.	.	.	.	A	.	.	.	.
.	.	.	.	.	.	.	.	H	.
.	.	.	.	.	.	.	.	.	.
.	.	.	.	A	A	.	.	.	A
.	.	.	.	A	.	.	.	.	.
.	.	.	.	.	.	.	.	.	.
.	.	.	.	H	.	.	.	.	.
.	.	P	.	.	.	.	.	.	.
.	.	.	.	.	.	.	H	.	.

Number of agents : 8
Number of deaths : 19
Number of births : 12
Number of iterations : 30
.	.	.	.	.	.	.	.	.	.
.	.	.	.	.	.	.	.	.	.
.	.	.	.	.	A	.	H	.	.
.	.	.	.	.	.	.	.	.	.
.	.	.	A	A	A	.	.	.	.
.	.	.	.	.	.	.	.	.	A
.	.	.	.	A	.	.	.	.	.
.	.	.	.	.	.	.	.	.	.
.	.	.	.	H	.	.	.	.	.
.	.	P	.	.	.	.	.	H	.

Number of agents : 9
Number of deaths : 19
Number of births : 13
Number of iterations : 31
.	.	.	.	.	.	.	.	.	.
.	.	.	.	.	.	.	.	.	.
.	.	.	.	.	.	.	.	H	.
.	.	.	.	.	A	.	.	.	.
.	.	.	A	A	A	.	.	.	A
.	.	.	.	A	.	.	.	.	.
.	.	.	.	.	.	.	.	.	.
.	.	.	.	.	.	.	.	.	.
.	.	.	.	.	.	.	.	.	.
.	.	.	.	H	.	.	.	.	H

Number of agents : 8
Number of deaths : 20
Number of births : 13
Number of iterations : 32
\end{verbatim}

On observe de plus la reproduction entre deux Aliens qui donne naissance à un nouvelle Alien sur le terrain de jeu.

On peut comparer cette évolution entre trois factions d'individu par rapport à un cycle Prédateur-Proie ou considérer l'évolution globale de la population et la comparer à l'évolution d'une population humaine.
\cleardoublepage

\section*{Conclusion}
\addcontentsline{toc}{section}{Conclusion}
Ce TP m'a pris beaucoup de temps et m'a passionné. Malheureusement le départ de mon binôme a alourdi ma charge de travail et je n'ai pas réussie à finaliser le TP et son rapport à temps.
Ce TP m'a néammoins fait découvrir beaucoup d'outils tel que les patrons, la librairie Boost et le gestionnaire CMake. Deplus, le projet a été hébergé sur un serveur svn.

La structure du TP défini en amont du projet m'a clairement facilité la tâche dans la gestion et l'implémentation de nouvelles fonctionnalités. Le patron stratégie définie pour la gestion des comportements n'avait pas été implémenté de cette manière à la base : toutes les methodes étaient écrites dans le \verb$GameControler$. La transposition de ces methodes dans le patron défini par \verb$IBehaviour$ a pris du temps mais n'est pas rentré en conflit avec la structure globale. Néanmoins cela a allégé le code et a séparé des algorithmes qui se répétés et qui pouvaient porter certaines confusions sur la lisibilité du code.
\cleardoublepage

\end{document}