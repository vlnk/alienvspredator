Les méthodes développées en \textbf{Recherche Opérationnelle} dépendent très souvent de plusieurs \textbf{paramètres} tels que le nombres maximum d'itérations, la précision numérique, \textit{etc}. Trouver la meilleure combinaison de valeurs pour ces paramètres est une étape complexe et coûteuse en temps, surtout lorsque les influences sont corrélées. 

Un logiciel nommé \textit{\textbf{\gls{NOMAD}}} permet d'optimiser ces paramètres tout en optimisant le temps de calcul. Pourtant ce lociciel a une interface assez rigide qui demande aux utilisateurs quelques efforts avant de pouvoir l'utiliser efficacement. Le but de ce projet est donc de réaliser un script écrit en \textbf{Python} afin d'\textbf{interfacer} le solveur \textit{\gls{NOMAD}} et le programme de Recherche Opérationnelle.

Le programme développé en Recherche Opérationnelle est présenté comme une \textbf{boîte noire} \textit{c'est-à-dire} que ni le solveur, ni le script ne peut accéder à son code source et seule son execution pourra permettre de définir les paramètres adéquat à l'utilisation du solveur. Le script analyse alors l'execution de la boîte noire afin d'en définir les paramètres qui seront stockés au \textbf{format \gls{XML}}. Puis ces paramètres seront envoyés au solveur afin d'optimiser l'execution de la boîte noire. Cette interface entre la boîte noire et le solveur permet alors de configurer ce dernier facilement sans que l'utilisateur le configure lui même.

La boîte noire devra intégré une execution \textit{"spéciale"} afin de permettre au script de configurer correctement le solveur. Le résultat final est que l'execution du script sur la boîte noire configure directement le solveur et peut même le lancer à plusieurs reprises au choix de l'utilisateur afin d'optimiser au mieux son problème de Recherche Opérationnelle. L'utilisateur peut alors \textbf{automatiser} ces calculs avec le solveur grâce au script. Cependant, ces résultats sont peu satisfaisant : le solveur \gls{NOMAD} récupère les paramètres d'une façon très stricte qu'il est dificile de respecter, ce qui rend le script encore inutilisable. De plus les paramètres récupérés sur la boîte noire dépendent fortement de cette boîte noire, si l'exécution \textit{spéciale} n'est pas réalisée, le script ne peut pas marcher. Ce qui limite d'autant plus son usage.  

\vskip40mm

\textbf{Mots clés :} Recherche Opérationnelle, paramètres, solveur \gls{NOMAD}, Python, boîte noire, interfaçage, \gls{XML}, automatisation.