Ce TP m'a pris beaucoup de temps et m'a passionné. Malheureusement le départ de mon binôme a alourdi ma charge de travail et je n'ai pas réussie à finaliser le TP et son rapport à temps.
Ce TP m'a néammoins fait découvrir beaucoup d'outils tel que les patrons, la librairie Boost et le gestionnaire CMake. Deplus, le projet a été hébergé sur un serveur svn.

La structure du TP défini en amont du projet m'a clairement facilité la tâche dans la gestion et l'implémentation de nouvelles fonctionnalités. Le patron stratégie définie pour la gestion des comportements n'avait pas été implémenté de cette manière à la base : toutes les methodes étaient écrites dans le \verb$GameControler$. La transposition de ces methodes dans le patron défini par \verb$IBehaviour$ a pris du temps mais n'est pas rentré en conflit avec la structure globale. Néanmoins cela a allégé le code et a séparé des algorithmes qui se répétés et qui pouvaient porter certaines confusions sur la lisibilité du code.